%-------------------------
% Resume in Latex
% Author : Sourabh Bajaj
% License : MIT
%------------------------

\documentclass[letterpaper,11pt]{article}

\usepackage{xcolor}
\definecolor{cBlue}{HTML}{455d7a}
\definecolor{cRed}{HTML}{e23e57}

\usepackage[empty]{fullpage}
\usepackage{titlesec}
\usepackage{verbatim}
\usepackage{enumitem}
\usepackage{hyperref}
\hypersetup{%
    colorlinks=true,
    urlcolor = cBlue
}
% \usepackage{fancyhdr}
\usepackage{microtype}

\usepackage[adobe-utopia]{mathdesign}
% \usepackage[T1]{fontenc}
\usepackage{fontawesome}

% \pagestyle{fancy}
% \fancyhf{} % clear all header and footer fields
% \fancyfoot{}
% \renewcommand{\headrulewidth}{0pt}
% \renewcommand{\footrulewidth}{0pt}

% Adjust margins
% \addtolength{\oddsidemargin}{-0.5in}
% \addtolength{\evensidemargin}{-0.5in}
% \addtolength{\textwidth}{1in}
% \addtolength{\topmargin}{-.5in}
% \addtolength{\textheight}{1.0in}
\usepackage[margin=0.75in]{geometry}

\urlstyle{same}

\raggedbottom
\raggedright
\setlength{\tabcolsep}{0in}

% Sections formatting
\titleformat{\section}{
    \vspace{-4pt}\scshape\raggedright\large\color{cRed}
}{}{0em}{}[\titlerule \vspace{-5pt}]

%-------------------------
% Custom commands
\newcommand{\resumeItem}[2]{
  \item\small{
    \textbf{#1}{: #2 \vspace{-2pt}}
  }
}

\newcommand{\resumeSubheading}[4]{
  \vspace{-1pt}\item
    \begin{tabular*}{0.97\textwidth}[t]{l@{\extracolsep{\fill}}r}
      \textbf{#1} & #2 \\
      \textit{\small#3} & \textit{\small #4} \\
    \end{tabular*}\vspace{-5pt}
}

\newcommand{\resumeSubItem}[2]{\resumeItem{#1}{#2}\vspace{-4pt}}

\renewcommand{\labelitemii}{$\circ$}

\newcommand{\resumeSubHeadingListStart}{\begin{itemize}[leftmargin=*]}
\newcommand{\resumeSubHeadingListEnd}{\end{itemize}}

\newcommand{\resumeItemListStart}{\begin{itemize}}
\newcommand{\resumeItemListEnd}{\end{itemize}\vspace{-5pt}}



%-------------------------------------------
%%%%%%  CV STARTS HERE  %%%%%%%%%%%%%%%%%%%%%%%%%%%%


\begin{document}

%----------HEADING-----------------
\begin{tabular*}{\textwidth}{l@{\extracolsep{\fill}}r}
    \textbf{\Huge{Kevin Lui Ph.D.}} 
    & \faEnvelope: \url{kevinywlui@gmail.com} \\

    \faGithub: \url{https://github.com/kevinywlui}
    &
    \faGlobe: \url{https://kevinlui.org/pages/code}
    \\

    \faLinkedin: \url{https://www.linkedin.com/in/kevin-lui-math/}
    &
    \faMapO: Seattle, WA
    \\

    Citizenship: USA
    &
\end{tabular*}


% --------PROGRAMMING SKILLS------------
\section{Technical Skills}
  \resumeSubHeadingListStart
    % \item{
    %         \textbf{Fluent in: }{Python, Git}
    % }
% \item{\vspace{-0.15cm}
    %         \textbf{Prior projects in: }{SQL, Bash, C++, MATLAB, Cython}
    % }
  \item{Python, Git, SQL, Bash, C++, MATLAB, Cython}
  \resumeSubHeadingListEnd
  \vspace{-0.5cm}

%-----------PROJECTS-----------------
\section{Select Personal Projects}
\resumeSubHeadingListStart
\resumeSubItem{langclass}{A model for determining the programming language from
    its source code. Model uses feature hashing of character-level 2-grams
    followed by a gradient boosting tree classifier. Live-version deployed on AWS
    in a Docker container running Flask/Waitress. See
    \url{http://langclass.kevinlui.org}.
}
\resumeSubItem{caleb}{Python package that helps with Latex citation by
automatically retrieving bibliographic information from publicly available
online sources. Used travis for CI, pytest for testing, and poetry for
dependency management. See \url{https://github.com/kevinywlui/caleb}}
% \resumeSubItem{Sagemath -- isomorphism testing}{Implemented isomorphism testing of
% modular abelian varieties into the Sagemath Python library. Currently in the
% process of being merged. See \url{https://trac.sagemath.org/ticket/28275}}
% \resumeSubItem{zlong\_alert.zsh}{Zsh plugin for alerting you upon the
% completion of a long-running command. See
% \url{https://github.com/kevinywlui/zlong_alert.zsh}}

\resumeSubHeadingListEnd


%-----------EXPERIENCE-----------------
\section{Experience}
\resumeSubHeadingListStart

\resumeSubheading
{University of Washington}{Seattle, WA}
{Researcher in Computational Number Theory}{Sep. 2014 --- Jun. 2019}
\resumeItemListStart
\resumeItem{Overview}
{
    % Thesis research centered around creating and implementing algorithms for
    % computing invariants of modular abelian varieties. I was able to create
    % algorithms for computing certain invariants where existing methods where
    % computationally infeasible or non-existent.
    Thesis research centered around creating and implementing algorithms in
    number theory. Created polynomial-time algorithms for computing invariant
    where existing methods where computationally infeasible or non-existent.
}
\resumeItem{Technical skills used}{
    % Code was written using the Sagemath Python library. Experiments were done
    % using Jupyter notebooks. Tables of related invariants were computed and
    % stored in PostgreSQL databases.
    Implemented algorithms using thousands of lines of Python. Invariants were
    computed in parallel on a 24-core server and stored in a PostgreSQL
    database. 
}
\resumeItem{Link to thesis}{\url{https://kevinlui.org/pages/thesis}}
\resumeItemListEnd

\resumeSubheading
{Sagemath Open Source Project}{Online}
{Volunteer Developer}{Jun. 2016 --- Present}
\resumeItemListStart
\resumeItem{Overview}
{Active contributor and user of Sagemath which is a Python mathematics
package, similar to scipy, suitable for research-level number theory
computations. See \url{https://www.sagemath.org/}}
\resumeItem{Contribution stats}{Authored 25 tickets, 19 accepted. Reviewed 10
tickets. Thousands of Python lines added.}
\resumeItem{UW Sagemath Seminar Organizer}{Helped 6 math
graduate students make their first open-source contribution!}
\resumeItem{Link to
code contributions and code reviews}{\url{https://kevinlui.org/pages/code\#sagemath}}
\resumeItemListEnd

\resumeSubheading
{Google Summer of Code - Sagemath}{Online}
{Student Developer}{Summer 2016}
\resumeItemListStart
\resumeItem{Overview}
{
    Implemented number theory algorithms in Python to help close
    feature gap between Sagemath and its closed-source competitor MAGMA.
}
\resumeItem{Outcome}{This code has been merged into the master branch
\url{https://trac.sagemath.org/ticket/21496} and is the foundation for my Ph.D.
thesis work.}
\resumeItemListEnd

\resumeSubheading
{UC Santa Barbara}{Santa Barbara, CA}
{Undergraduate Summer Researcher}{Summer 2012}
\resumeItemListStart
\resumeItem{Overview}{Derived a parametric model for determining a consensus
given a group of experts' rankings.}
\resumeItem{Role}{Wrote MATLAB code to solve the LP problem derived in the
paper. Worked closely with faculty mentor to develop the model.}
\resumeItem{Outcome}{Published:
\url{https://www.sciencedirect.com/science/article/pii/S0165011413003308}}
\resumeItemListEnd


\resumeSubHeadingListEnd

%-----------EDUCATION-----------------
\section{Education}
\resumeSubHeadingListStart
\resumeSubheading
{University of Washington, Seattle}{Seattle, WA}
{Ph.D. in Mathematics specializing in Computational Number Theory}{June 2019}
% \resumeSubheading
% {University of Washington, Seattle}{Seattle, WA}
% {Masters in Mathematics}{June 2018}
\resumeSubheading
{University of California, Santa Barbara}{Santa Barbara, CA}
{Bachelors in Mathematics}{June 2014}
\resumeSubHeadingListEnd


%-------------------------------------------
\end{document}
