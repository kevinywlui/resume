%-------------------------
% Resume in Latex
% Author : Sourabh Bajaj
% License : MIT
%------------------------

\documentclass[letterpaper,10pt]{article}

\usepackage{xcolor}
\definecolor{cBlue}{HTML}{455d7a}
\definecolor{cRed}{HTML}{e23e57}

\usepackage[empty]{fullpage}
\usepackage{titlesec}
\usepackage{verbatim}
\usepackage{enumitem}
\usepackage{hyperref}
\hypersetup{
    colorlinks=true,
    urlcolor = cBlue
}
% \usepackage{fancyhdr}
\usepackage{microtype}

\usepackage[adobe-utopia]{mathdesign}
% \usepackage[T1]{fontenc}
\usepackage{fontawesome}

% \pagestyle{fancy}
% \fancyhf{} % clear all header and footer fields
% \fancyfoot{}
% \renewcommand{\headrulewidth}{0pt}
% \renewcommand{\footrulewidth}{0pt}

% Adjust margins
\addtolength{\oddsidemargin}{-0.5in}
\addtolength{\evensidemargin}{-0.5in}
\addtolength{\textwidth}{1in}
\addtolength{\topmargin}{-.5in}
\addtolength{\textheight}{1.0in}

\urlstyle{same}

\raggedbottom
\raggedright
\setlength{\tabcolsep}{0in}

% Sections formatting
\titleformat{\section}{
    \vspace{-4pt}\scshape\raggedright\large\color{cRed}
}{}{0em}{}[\titlerule \vspace{-5pt}]

%-------------------------
% Custom commands
\newcommand{\resumeItem}[2]{
  \item\small{
    \textbf{#1}{: #2 \vspace{-2pt}}
  }
}

\newcommand{\resumeSubheading}[4]{
  \vspace{-1pt}\item
    \begin{tabular*}{0.97\textwidth}[t]{l@{\extracolsep{\fill}}r}
      \textbf{#1} & #2 \\
      \textit{\small#3} & \textit{\small #4} \\
    \end{tabular*}\vspace{-5pt}
}

\newcommand{\resumeSubItem}[2]{\resumeItem{#1}{#2}\vspace{-4pt}}

\renewcommand{\labelitemii}{$\circ$}

\newcommand{\resumeSubHeadingListStart}{\begin{itemize}[leftmargin=*]}
\newcommand{\resumeSubHeadingListEnd}{\end{itemize}}

\newcommand{\resumeItemListStart}{\begin{itemize}}
\newcommand{\resumeItemListEnd}{\end{itemize}\vspace{-5pt}}



%-------------------------------------------
%%%%%%  CV STARTS HERE  %%%%%%%%%%%%%%%%%%%%%%%%%%%%


\begin{document}

%----------HEADING-----------------
\begin{tabular*}{\textwidth}{l@{\extracolsep{\fill}}r}
    \textbf{\Huge{Kevin Lui Ph.D.}} 
    & \faEnvelope: \url{kevinywlui@gmail.com} \\
    \faGlobe: \url{https://kevinlui.org}
    & \faGithub: \url{https://github.com/kevinywlui} \\
    \faMapO: Seattle, WA &
    Citizenship: USA
\end{tabular*}


%-----------EDUCATION-----------------
\section{Education}
\resumeSubHeadingListStart
\resumeSubheading
{University of Washington, Seattle}{Seattle, WA}
{Ph.D. in Mathematics specializing in Computational Number Theory under William
Stein}{June 2019}
% \resumeSubheading
% {University of Washington, Seattle}{Seattle, WA}
% {Masters in Mathematics}{June 2018}
\resumeSubheading
{University of California, Santa Barbara}{Santa Barbara, CA}
{Bachelors in Mathematics}{June 2014}
\resumeSubHeadingListEnd


%-----------EXPERIENCE-----------------
\section{Experience}
\resumeSubHeadingListStart

\resumeSubheading
{University of Washington}{Seattle, WA}
{Researcher in computational number theory}{Sep. 2014 --- Jun. 2019}
\resumeItemListStart
\resumeItem{Overview}
{
    Thesis research centered around creating and implementing algorithms for
    computing invariants of modular abelian varieties. These are objects of
    great number theoretic importance. The overarching theme is bring these
    abstract objects into a linear algebraic setting that is more suitable for
    computations.}
\resumeItem{Technical skills used}{
    Code was written using the Sagemath Python library. Experiments were done
    using Jupyter notebooks. Tables of related invariants were computed in
    parallel and stored in PostgreSQL databases.
}
\resumeItem{Link to thesis}{\url{https://kevinlui.org/pages/thesis}}
\resumeItemListEnd

\resumeSubheading
{Sagemath Open Source Project}{Online}
{Active Contributor and User}{Jun. 2016 --- Present}
\resumeItemListStart
\resumeItem{Overview}
{Actively contribute to and use Sagemath which is a Python mathematics
package, similar to scipy, suitable for research-level number theory
computations.}
\resumeItem{Link to
contributions}{\url{https://kevinlui.org/pages/code\#sagemath}}
% \resumeItem{Google Summer of Code 2016}{Improved functionality of modular
%     abelian varieties in Sagemath. This work would eventually evolve into my
%     Ph.D. thesis work. Development was done over Github:
%     \url{https://github.com/williamstein/sage_modabvar} and has since been
%     merged into the Sagemath master branch.}
\resumeItem{Sage days 87}{Attended a workshop on $p$-adic number functionality
in Sagemath. Worked on re-basing an old branch on elliptic curves.}
\resumeItemListEnd

\resumeSubheading
{Google Summer of Code - Sagemath}{Online}
{Student Developer}{Summer 2016}
\resumeItemListStart
\resumeItem{Overview}
{Implemented algorithms related to modular abelian
varieties in Python. This required translating algorithms found in research
papers to something implementable within the Sagemath library.} 
\resumeItem{Outcome}{This code has been merged into the master branch
\url{https://trac.sagemath.org/ticket/21496} and has morphed into my Ph.D.
thesis topic.}
\resumeItemListEnd


\resumeSubheading
{University of Washington Sage Seminar}{Seattle, WA}
{Organizer}{Jun. 2019 --- Present}
\resumeItemListStart
\resumeItem{Overview}{Mentoring a group of math graduate students towards
contributing to the Sagemath open source projection.}
\resumeItem{Primary role}{Teaching the Sagemath development process which
involves building Sagemath, collaborating via Sagemath's git-trac server, and
following Sagemath developer conventions.}
\resumeItem{Secondary role}{Bringer of coffee and Linux tech support.}
\resumeItem{Success}{About 6 attendees have made their first open source code
contribution.}
\resumeItemListEnd

% \resumeSubheading
% {University of Washington}{Seattle, WA}
% {Teaching Assistant/Instructor}{Sep. 2014 - Present}
% \resumeItemListStart
% \resumeItem{Instructor - Linear Algebra}
% {Primary instructor for introductory (5 times) and advanced (1 time) linear
%     algebra classes. Created my own lectures and exams.
% }
% \resumeItem{Washington Experimental Mathematics Lab Graduate Mentor}
% {Helped 3 undergraduate students use Python to investigate tilings of the real
% plane.}
% \resumeItem{Teaching Assistant - Calculus}
% {Answered homework questions in biweekly quiz sections and office hours.}
% \resumeItem{Grader}
% {Grader for the upper-division topology class. Graded weekly homework and held
% office hours.}
% \resumeItemListEnd


\resumeSubHeadingListEnd

\section{Programming Coursework}
All taken as an undergraduate at UC Santa Barbara. Primarily used C++.

$\bullet$ Data Structures and Algorithms, Formal
            Languages and Automata, Cryptography, Computer Theorem Proving, Logic


%-----------PROJECTS-----------------
\section{Projects}
\resumeSubHeadingListStart
\resumeSubItem{Links}{\url{https://kevinlui.org/pages/code/}}
\resumeSubItem{caleb}{Python package that helps with Latex citation by
automatically retrieving bibliographic information from publicly available
online sources. Using this to learn CI, pytest, and making a package available
on pypi.}
\resumeSubItem{Sagemath -- isomorphism testing}{Implemented isomorphism testing of
modular abelian varieties into the Sagemath Python library. Currently in the
process of being merged: \url{https://trac.sagemath.org/ticket/28275}}
\resumeSubHeadingListEnd


% --------PROGRAMMING SKILLS------------
\section{Programming Skills}
  \resumeSubHeadingListStart
    \item{
            \textbf{Proficient: }{Python, Sagemath, Latex, Git}
    }
\item{\vspace{-0.15cm}
            \textbf{Prior experience: }{PostgreSQL, SQLite, Bash, GNU/Linux,
            C++, vim, MATLAB/Octave}
    }
  \resumeSubHeadingListEnd


%-------------------------------------------
\end{document}
